\setchapterstyle{kao}
\chapter{Conclusions and Future Developments}
\labch{conclusions_future_developments}

In conclusion, the extensive validation across ten different downstream tasks,
as discussed in~\refch{anatcl}, has demonstrated that integrating anatomical
information during training enhances the accuracy of predictions for various
neurological and psychiatric conditions. The findings also indicate a partial
improvement in the accuracy for clinical assessment scores and phenotypes. These
results suggest that enriching these learning methods with anatomical data can
yield more robust and generalizable models applicable to a range of downstream
tasks. The insights gained from these models could be crucial in developing
personalized treatment plans for patients. By accurately characterizing the
neurological basis of various psychiatric and neurodegenerative disorders, these
models could significantly influence the design of tailored therapeutic
interventions. Additionally, achieving higher accuracy in detecting biomarkers
such as brain age from these models could significantly enhance the diagnosis of
specific neurological disorders, thereby improving patient outcomes. 

However, further research is required to refine and enhance these methods. For
example, an obvious limitation of this approach is that these methods still rely
on the age attribute, which may not be available in all neuroimaging datasets.
Further research in this area could lead to the development of novel methods
that rely solely on anatomical measures for weak supervision, potentially
achieving superior performance compared to fully self-supervised methods. Since
anatomical measures can be automatically extracted using established
neuroimaging algorithms, these innovative methods could essentially be
considered fully self-supervised.

Furthermore, this research primarily utilized a specific brain atlas, thereby
not incorporating the range of other atlases available in the neuroimaging
literature. Future work could involve exploring additional atlases that utilize
different anatomical measures. Such an endeavor would necessitate a meticulous
comparative evaluation of these alternative atlases to determine their efficacy
and accuracy relative to the one used in this study. Expanding the scope to
include a variety of atlases could enhance the robustness and applicability of
the findings, potentially offering a more comprehensive understanding of brain
anatomy and its implications for neuroimaging analysis.

Another potential direction not explored in this thesis involves utilizing other
acquisition methods as additional learning data. The aim would be to integrate
diverse neuroimaging modalities (sMRI, fMRI, dMRI), signals (EEG) and other
relevant data (clinical assessment scores, phenotypic data) into the learning
process of the model. The idea is to create a more holistic model that
incorporates a broader range of modalities in order to provide a more
comprehensive understanding of brain health and pathology. The integration of
diverse sources of information could be achieved either by incorporating it into
the loss formulation, as examined in this work, or by developing a novel
multimodal model architecture capable of extracting and combining salient
information from input data.
Recent research~\sidecite{radford_clip_2021} has also demonstrated the
feasibility of augmenting deep learning models not only with imaging data but
also with textual data. For instance, in the medical domain, there has been a
research effort~\sidecite{wang_medclip_2022} directed towards integrating
clinical records and assessment scores. This research direction is further
supported by recent works~\sidecite{venugopalan_multimodal_2021} that
demonstrated by integrating data of different nature such as EEG and clinical
assessments could significantly enrich the representations learned by the model.

The findings from this research direction could pave the way for the development
of large multimodal models that could serve as foundational tools in
personalized medicine. Employing the principles of transfer learning, these
models can be adapted for personalized predictions of psychiatric conditions,
such as Autism Spectrum Disorder (ASD), and neurodegenerative conditions, such
as Alzheimer's Disease (AD). To achieve this, multimodal and/or longitudinal
data can be utilized to create detailed, patient-specific profiles and to model
the progression of various conditions. Such personalized models can then offer
customized insights and treatment recommendations, thereby enhancing patient
outcomes by accommodating the unique variations in brain structure and function
specific to each individual.

\section*{Ethical Considerations}
In the development and experiments described within this thesis, strict
adherence to ethical standards was maintained. A significant aspect of these
ethical considerations involves the use of data. It is important to note that
there were no ethical issues concerning the datasets utilized, as all were
publicly accessible and composed of previously anonymized patient data, ensuring
that individual privacy was preserved and that the data could be used without
compromising patient confidentiality.

Furthermore, the applications of the findings discussed in this research thesis
are aimed at enhancing the accuracy of diagnoses and the identification of
pathological conditions related to the brain. By improving diagnostic
capabilities through more precise imaging analysis, the potential for positive
impacts on patient outcomes is significant. The use of advanced machine learning
models in medical imaging can lead to earlier detection of diseases, more
tailored treatment plans, and ultimately, better patient care. This alignment
with the goal of improving healthcare outcomes further supports the ethical
justification for this research.

\section*{Declaration of Originality}
I declare to be responsible for the content I'm presenting in order to
obtain the final degree, not to have plagiarized in all or part of, the work
produced by others and having cited original sources in consistent way with
current plagiarism regulations and copyright. I am also aware that in case of
false declaration, I could incur in law penalties and my admission to final exam
could be denied.