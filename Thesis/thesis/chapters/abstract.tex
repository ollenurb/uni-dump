\chapter*{Abstract}
\addcontentsline{toc}{chapter}{Abstract}  

In recent years, the application of deep neural networks, particularly
Convolutional Neural Networks (CNNs), has yielded remarkable results across
various domains that benefit from automated image analysis. The field of
neuroimaging has particularly benefited from these advances, driven by the
growing availability of brain imaging data, facilitated by enhancements in
non-invasive acquisition methods like magnetic resonance imaging. 

However, the application of deep learning in neuroimaging faces challenges,
especially when applied to the prediction of psychiatric and neurological
disorders, which encompasses a wide range of clinical, biological, and
environmental factors. This challenge is further exacerbated by data scarcity.
Although the volume of neuroimaging data has increased in recent years due to
collaborative research efforts, these datasets are often limited in size,
especially when focused on specific neurological conditions. To address these
challenges, Transfer Learning and Contrastive Learning have emerged as effective
strategies, showing good performances in various neuroimaging tasks compared to
traditional machine learning approaches. According to this framework, a model is
initially pre-trained on a large dataset of healthy subjects using Contrastive
Learning techniques. Subsequently, this pre-trained model is fine-tuned for a
specific task using a smaller cohort of patients, typically associated with a
particular condition or phenotype.

Neuroimaging datasets are also rich in additional patient information, such as
age, sex, and other neuroanatomical data. These features are valuable markers,
particularly when correlated with data derived from neuroimaging studies. Recent
research efforts have focused on integrating these features into the
pre-training phase using contrastive learning techniques. However, current
state-of-the-art methods predominantly rely on chronological age as the primary
feature during pre-training, which may not sufficiently capture the complex
information inherent in brain MRI data.

This thesis introduces a novel approach designed to overcome this limitation.
The core of this research involves the development and application of a new
Contrastive Learning method termed AnatCL, which integrates multiple anatomical
measures derived from brain MRIs along with demographic data (patient age). By
incorporating additional features, AnatCL facilitates the learning of more
meaningful and generalizable representation spaces that more accurately reflect
individual variability and aging patterns. Results from evaluating various
downstream tasks across multiple neuroimaging datasets suggest that enriching
these learning methods with additional data can yield more robust and
generalizable models.